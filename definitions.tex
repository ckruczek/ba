\pagestyle{fancyplain}

\fancypagestyle{plain}{%
  \fancyhf{}%
  \fancyfoot[C]{\thepage}%
  \renewcommand{\headrulewidth}{0pt}% Line at the header invisible
  \renewcommand{\footrulewidth}{1.5pt} 
}

\newcounter{vicCounter}
\setcounter{vicCounter}{1}
\newcommand{\mytitle}{Entwicklung eines Minibetriebssystems auf Basis eines ARM926 Prozessors}
\newcommand{\mops}{\textbf{\textit{{\large M}\textsc{op}{\large S} }}}
\newcommand{\me}{Christopher \textsc{Kruczek}}

\renewcommand{\chaptermark}[1]{\markboth{#1}{}}
\renewcommand{\sectionmark}[1]{\markright{#1}}
\renewcommand{\footrulewidth}{1.5pt}
\renewcommand{\arraystretch}{1.5}

\lhead[\fancyplain{}{\leftmark}]         {\fancyplain{}{\leftmark}}
\rhead[\fancyplain{}{\rightmark}]       {\fancyplain{}{\rightmark}}
\cfoot[\fancyplain{\thepage}]  {\fancyplain{\thepage}{\thepage}}

% Bjornstrup chapter formating
\ChNumVar{\fontsize{76}{80}\usefont{OT1}{pzc}{m}{n}\selectfont}
\ChTitleVar{\raggedright\Huge\bf}

\renewcommand\lstlistlistingname{Quellcode-Ausschnitte}

%==============================================================================
%																		Listings
%==============================================================================
\definecolor{cred}{rgb}{0.6,0,0} % for strings
\definecolor{cgreen}{rgb}{0.25,0.5,0.35} % comments
\definecolor{background}{RGB}{242,242,242}
\definecolor{cpurple}{rgb}{0.5,0,0.35} % keywords
\definecolor{cdocblue}{rgb}{0.25,0.35,0.75} % c doc

\RequirePackage{listings} 

\lstset{language=C} 
\lstset{keywordstyle=\color{cpurple}\bfseries} 
\lstset{breaklines=true} 
\lstset{basicstyle=\scriptsize\ttfamily}%\ttfamily\fontsize{10}{10}} 
\lstset{xleftmargin=5px}
\lstset{tabsize=2}
\lstset{frameround=ffff}
\lstset{frame=single}
\lstset{framerule=0.5pt}
\lstset{backgroundcolor=\color{background}}
\lstset{numbers=left}
\lstset{showstringspaces=false}
\lstset{stringstyle=\color{cred}}
\lstset{commentstyle=\color{cgreen}}
\lstset{morecomment=[s][\color{cdocblue}]{/**}{*/}}


\renewcommand{\lstlistingname}{Code-Beispiel}
\DefineBibliographyStrings{german}{%
urlseen = {Letzter Zugriff},
}