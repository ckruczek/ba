\begin{quote}
\blockquote{\textit{Der Entwurf eines Betriebssystem erfordert eher ein ingeneursm\"a\ss iges Vorgehen, als ein exaktes wissenschaftliches. Es ist schwieriger, klaere Ziele zu definieren und diese zu erreichen.}}\parencite[911]{os}
\end{quote}
Mit dieser Aussage leitet Tanenbaum das Thema der Entwicklung eines Betriebssystem ein. Und genau mit dieser Frage soll diese Bachelorarbeit eingeleitet werden. Was sollen also die Ziele dieser Bachelorarbeit sein?\\\\
Die Entwicklung eines Betriebssystems bringt viele Schwierigkeiten und Herausforderungen mit sich. Um der n\"achsten Generation von Informatikern einen leichten Einstieg in dieses Thema zu bieten ist diese Bachelorarbeit entstanden. Sie dient als Anschauungsmaterial und besch\"aftigt sich mit den Grundlagen der Betriebssystementwicklung in Embedded Systemen. Das Betriebssystem, mit dem Akronymnamen \mops - \textbf{M}ini \textbf{Op}erating \textbf{S}ystem, soll seinen Haupteinsatzzweck im Lehrbereich an der HTW-Berlin, f\"ur den Studiengang Angewandte Informatik und Wirtschaftsinformatik, finden. \\
Es dient als Basis und zur Weiterentwicklung weiterer Module, die im Rahmen des Projektes \textbf{FOCOS} unter Herr Prof. Dr. Burkhard Messer enstehen werden, und soll zeigen wie man mit relativ wenig Aufwand eine fundierte Grundlage zur Betriebssystementwicklung erstellen kann. Es werden Grundlegende Probleme wie der Entwurf eines Betriebssystems, die Erstellung von Exceptionhandlern, es wird die Entwicklung von IRQ- und FIQ-Handlern diskutiert und wichtige Themen wie Prozessverwaltung und Scheduling beleuchtet.\\
Da dieses Projekt im Rahmen einer Lehrveranstaltung als Anschauungsmaterial dienen soll, wird besonderer Wert auf Quellcode und Grafische Untermalung im Entwurf gelegt. Zum Abschluss noch ein Zitat von Fernando Corbat\'o, einem der Entwickler von CTSS\footnote{Compatible Time Sharing System} und MULTICS\footnote{Multiplexed Information and Computing Service}
\begin{quote}
\blockquote{\textit{\ldots Meine Definition von Eleganz ist das Erreichen einer gegebenen Funktionalit\"at mit einem Minimum an Mecuhanismen und einem Maximum an Klarheit.}}\parencite[915]{os}
\end{quote}
Dieses Zitat soll begleitend f\"ur diese ganze Bachelorarbeit stehen.