\begin{quote}
\blockquote{\textit{Der Entwurf eines Betriebssystem erfordert eher ein ingeneursm\"a\ss iges Vorgehen, als ein exaktes wissenschaftliches. Es ist schwieriger, klaere Ziele zu definieren und diese zu erreichen.}}\parencite[911]{os}
\end{quote}
Mit dieser Aussage leitet Tanenbaum das Thema der Entwicklung eines Betriebssystem ein. Und genau mit dieser Frage soll diese Bachelorarbeit eingeleitet werden. Was sollen also die Ziele dieser Bachelorarbeit sein?\\\\
Das Betriebssystem was im folgenden vorgestellt wird tr\"agt den Akronymnamen \mops - \textbf{M}ini \textbf{Op}erating \textbf{S}ystem und soll seinen Haupteinsatzzweck im Lehrbereich der HTW-Berlin, f\"ur den Studiengang Angewandte Informatik und Wirtschaftsinformatik, finden.\\ \\
Die Entwicklung eines Betriebssystems bringt viele Schwierigkeiten und Herausforderungen mit sich. Eine der Schwierigkeiten ist die Handhabung mit Embedded Systems. Hier zeigt \mops wie man mit einem verh\"altnism\"a\ss ig kleinen Prozessor eine stabile L\"osung erarbeiten kann. Eine weitere Schwierigkeit in diesem Bereich stellt die Benutzung der vorhandenen Hardware dar. Das hei\ss t z.B. wie wird der Timer konfiguriert, an welche Stelle im RAM muss er geladen oder wie k\"onnen die Ticks behandelt werden. Weiterhin werden Komponenten wie die Serielle Schnittstelle zwischen Tastatur und Monitor beleuchtet. Hier wirdu gezeigt wie man eine Eingabe von der Tastatur abfangen kann, welches Hardwareteil dazu konfiguriert werden muss und wie die Daten auf dem Monitor angezeigt werden k\"onnen. Aber auch der Punkt der Interrupts kommt in diesem Konzept nicht zu kurz. Was bedeuten Interrupts? Wie m\"ussen sie konfiguriert werden? Welche Interrupthandler werden ben\"otigt?\\ Neben all diesen Aspekten stellen Embedded Systems wie z.B. Mobil Telefone, Drucker, Kaffeemaschinen, Tablets den Entwickler vor gro\ss e Herausforderungen was die Thematik Prozessmanagment und Scheduling angeht. Deshalb werden diese Aspekte besonders beleuchtet. Das bedeudet es gibt tiefe Einblicke in das Thema - Laden einens Prozesses -, - Starten eines Prozesses -, - Wechsel zwischen den Prozessen - und vieles mehr. Um der n\"achsten Generation von Informatikern einen leichten Einstieg in dieses Thema zu bieten ist diese Bachelorarbeit entstanden. Sie dient als Anschauungsmaterial und besch\"aftigt sich mit den Grundlagen der Betriebssystementwicklung in Embedded Systemen. Zudem soll die Arbeit den zuk\"unftigen Projektteilnehmern die Angst vor der Betriebssystementwicklung nehmen. \\
Im Rahmen des Projektes \textbf{FOCOS - Family of configurated operating systems}, f\"ur das Sommersemester 2013 begleitet durch Prof. Dr. Messer, soll diese Bachelorarbeit als Basis zur Weiterentwicklung von neuen Komponenten dienen.\\
Da dieses Projekt im Rahmen einer Lehrveranstaltung als Anschauungsmaterial dienen soll, wird besonderer Wert auf Quellcode und Grafische Untermalung im Entwurf gelegt. Zum Abschluss noch ein Zitat von Fernando Corbat\'o, einem der Entwickler von CTSS\footnote{Compatible Time Sharing System} und MULTICS\footnote{Multiplexed Information and Computing Service}
\begin{quote}
\blockquote{\textit{\ldots Meine Definition von Eleganz ist das Erreichen einer gegebenen Funktionalit\"at mit einem Minimum an Mecuhanismen und einem Maximum an Klarheit.}}\parencite[915]{os}
\end{quote}
Dieses Zitat f\"uhrt uns zu einem weiteren wichtigen Punkt in dieser Arbeit. \mops ist auf Einfachheit und \"Uberschaubarkeit ausgelegt. Das hei\ss t es wurde Wert darauf gelegt keine unn\"otigen Sachen zu implementieren.