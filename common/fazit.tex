Ziel der vorliegenden Arbeit war es, wie in der Einleitung beschrieben, einen Entwurf eines Betriebssystem mit maximal notwendigen Funktionsumfang aber mit minimalem Aufwand zu erschaffen.\\
Es wurde die wichtigstens, f\"ur ein Lehrmaterial notwendigen, Mechanismen umgesetzt und anhand des Entwurfes ist ein klares Bild von \mops entstanden. Was den Entwurf betrifft konnte gezeigt werden das der Umfang einer minimalistischen Definition keine gro\ss e H\"urde darstellt. Andererseits musste festgestellt werden das die Umsetzung dieses Entwurf durchaus komplizierter war als anfangs angenommen wurde. Dieses Ergebnis konnte deshalb evaluiert werden, weil die Implementationsphase sich fast bis zum Ende der Bachelorarbeit erstreckte. Dennoch kann im Rahmen des Projektes festgehalten werden das es faktisch m\"oglich ist so ein Projekt zu stemmen. Dieses System kann also als Beispiel f\"ur die schnelle Implementation eines Betriebssystem gesehen werden.\\\\
Nat\"urlich musste man sich auch klar von einigen Features distanzieren. Dies betreffend sollen hier nur exemplarisch die Stichpunkte Speichermanagment, Datei-System und Multiprozessorunterst\"utzung genannt werden. Jedoch stellt sich auch die Frage wie es in der Zukunft mit \mops weitergehen soll und welche Features noch umgesetzt werden sollen. Hierzu ist gesagt das dieses Projekt definitiv weiterverfolgt wird und weiterhin von Prof. Dr. Messer, im Rahmen seines Projektes \textbf{FOCOS}, unterst\"utzt wird. Mit einer weiteren Version soll zun\"achst die Code-Basis aufger\"aumt und weiter Optimiert werden, so wie eine Unterst\"utzung zum starten von Prozessen zur Laufzeit, wie auch ein Speichermanagment.\\\\
Abschlie\ss enden kann gesagt werden das die Entwicklung eines Betriebssystem im Rahmen einer Bachelorarbeit ein sehr komplexes und schweres Unterfangen ist, andererseits aber einen umfangreichen Einblick und sehr viel Erfahrung mitbringt. Eine Person die ich hier sehr hervorheben will ist Prof. Dr. Burkhard Messer, dank seines unverwechselbaren Interesse und seines immer kompetenten und dauerhaften Einsatzes w\"are diese Arbeit mit sicherheit nicht zu diesem Ergebnis gekommen.