Ziel der vorliegenden Arbeit war es, wie in der Einleitung beschrieben, einen Entwurf eines Betriebssystemes mit maximal notwendigem Funktionsumfang, aber mit minimalem Aufwand, zu erschaffen.\\
Es wurden die wichtigsten, f\"ur ein Lehrmaterial notwendigen, Mechanismen umgesetzt und anhand des Entwurfes ist ein klares Bild von \mops entstanden. Was den Entwurf betrifft, konnte gezeigt werden, dass der Umfang einer minimalistischen Definition keine gro\ss e H\"urde darstellt. Andererseits musste festgestellt werden, dass die Umsetzung dieses Entwurf durchaus komplizierter war als Anfangs angenommen wurde. Dieses Ergebnis konnte deshalb evaluiert werden, weil die Implementationsphase sich fast bis zum Ende der Bachelorarbeit erstreckte. Dennoch kann im Rahmen des Projektes festgehalten werden, dass es faktisch m\"oglich ist, solch ein Projekt zu realisieren. Dieses System kann also als Beispiel f\"ur die schnelle Implementation eines Betriebssystemes angesehen werden.\\\\
Nat\"urlich musste man sich auch klar von einigen Features distanzieren. Dies betreffend sollen hier nur exemplarisch die Stichpunkte Speichermanagment, Datei-System und Multiprozessorunterst\"utzung genannt werden. Jedoch stellt sich auch die Frage, wie es in der Zukunft mit \mops weitergehen soll und welche Features noch umgesetzt werden sollen. Hierzu soll gesagt sein, dass dieses Projekt definitiv weiterverfolgt wird und weiterhin von Prof. Dr. Messer, im Rahmen seines Projektes \textbf{FOCOS - Family of Configurated Operating Systems}, unterst\"utzt wird. Mit einer weiteren Version soll zun\"achst die Code-Basis aufger\"aumt und weiter optimiert werden, eine Unterst\"utzung zum Starten von Prozessen zur Laufzeit und ein Speichermanagment angeboten werden.\\\\
Abschlie\ss end kann gesagt werden, dass die Entwicklung eines Betriebssystemes im Rahmen einer Bachelorarbeit ein sehr komplexe Aufgabenstellung darstellt, andererseits aber einen umfangreichen Einblick und sehr viele neue Erfahrungswerte mit sich bringt. Einen besonderen Dank m\"ochte ich an dieser Stelle Herr Prof. Dr. Burkhard Messer aussprechen. Er stand mir immer mit kompetenten Rat und Tat zu Seite. Ohne Ihn h\"atte das Projekt nicht diese Besonderen Ausma\ss e angenommen.
\nocite{clanguageII}