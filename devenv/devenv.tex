F\"ur die Entwicklung von ARM-basierten Komponenten kann man die Eclipse Entwicklungsumgebung nutzen. Ich entschied mich jedoch f\"ur den konservativen Weg und entwickle seit her, mit dem in Linux integrierten Editor \textit{vim}. 
Da mir zum Zeitpunkt des Entwicklungsbeginn keine physikalische Hardware zur Verf\"ugung stand entschied ich mich einen Emulator zu benutzen. Die erste Wahl viel hier auf den 
\href{http://infocenter.arm.com/help/index.jsp?topic=/com.arm.doc.dai0032f/index.html}{ARMulator}